\chapter{Introduction}
\label{chap1}

In recent years, we've seen a significant rise in the use of the internet and technology applications in our daily routines. As we increasingly rely on smartphones and other devices, they are making our lives more convenient, adaptable and in some cases, even safer. A vast array of applications and services are readily available to us, ranging from simple payment and ticket booking systems to high-accuracy GPS maps and numerous other innovative technologies\cite{madakam2015internet}. These technologies can be developed by both leading tech companies and smaller-scale initiatives. Often, these smaller companies or groups of developers aim to implement the well-established and ever-expanding applications of the Internet of Things network. This large network of connected devices is full of potential used to create new solutions that can meet many different needs. From creating smart environments that respond to user behavior\cite{AdvancesinCrowdAnalysis,OverviewofUrbanMobility}, to developing systems that monitor and manage resources for efficiency, such technological innovations are pushing the boundaries of what's possible with IoT technology.

The Internet of Things, commonly referred to as IoT, involves the interconnection of a vast array of devices and services via the internet, with the ultimate aim to orchestrate collective operation.\cite{madakam2015internet,rose2015internet}. This interconnection is designed to generate applications and internet solutions for a wide range of scenarios. Such applications can be found in smart homes, healthcare, agriculture, transportation, energy management, various types of automation and of course, environmental monitoring,\cite{Systematicreview,BIBRI2018230,7562698} which is the focus of our thesis. The broad scope of these implementations aims to enhance the quality of our lives, whether we're discussing large-scale applications or smaller companies seeking straightforward solutions that require internet interconnectivity. The rapid expansion of such applications, coupled with the vast open-source community, can lead to the development of solutions for a wide array of challenges. This is particularly beneficial in certain areas that require more expedient solutions to address their unique local problems. 

Given this local problem-solving approach and the accessibility of smart internet tools provided by IoT, we see the potential for creating widely available applications focused on environmental monitoring and air quality. Air quality and environmental pollution have been major concerns for modern societies and political spectrum in recent years\cite{harper2017environment}. The industrial revolution of the past century, along with the continued use of methods with high carbon and particulate matter footprints, have raised significant concerns, especially in recent years\cite{EuropeanEnvironmentAgency}. From a governmental perspective, particularly in Europe~\cite{EuropeEnvironment}, there has been significant effort to revolutionize industrial production methods to be more efficient and environmentally friendly. Despite these valuable and substantial efforts to reduce air pollution, the problem is far from being fully resolved. These circumstances lead the way for innovative ideas to monitor air quality and pollution in efficient and affordable ways, primarily making them widely available to people concerned about such issues. This is where the extensive capabilities of IoT applications come into play, providing the opportunity to effectively monitor the overall air pollution footprint at a specific location\cite{MobileAirPollutionMonitoring}.

\section{Thesis' Vision}
In our approach, we aim to create a highly interoperable and scalable solution that follows to the core principles of IoT. This is achieved by enhancing the capabilities of the SYNAISTHISI platform with emerging technologies and network protocols increasingly adopted across a broad range of technological infrastructures. SYNAISTHISI platform~\cite{Synaisthisi} initiated the contemporary concept of interconnecting diverse resources for various purposes, complying with the fundamental framework of IoT applications. Its basic idea was the creation of an IoT middle-ware that links a vast array of services, aiming to accomplish any given task: A service hosting platform where various system applications can be developed and operated by users in an easy and scalable manner. Also, the support of multiple protocols across different platforms, systems and services, makes the process of interconnecting elements like sensors, actuators and embedded systems more accessible. This combination of older protocols with newer ones enhances the scalability of our application and facilitates its accessibility for future upgrades and covers a wide spectrum of use cases.

In this regard, we try to bridge our concern and the fundamental right of everyone to access information about the air quality in their living environment. More specifically, there has been significant interest in PM\textsubscript{2.5} particles in the atmosphere, along with general factors like temperature and humidity and their long-term effects on human health\cite{effectofheatandairpollution,RecentInsightsParticulateMatter}. We aim to provide an easy-to-use service for everyone, but most importantly, to encourage technologically savvy individuals to explore the power of existing IoT and open-source infrastructures. With some insight, they can create their own services and conduct their own research on any given problem. This luxury of leveraging existing Cloud infrastructures\cite{ModernComputingParadigms} and adding services to them that can address a wide range of problems is relatively new. Our approach lays the groundwork for innovative methods of developing applications that effectively enhance our lives and keep us informed about existing situations, all while adhering to the principles of the IoT philosophy.

This thesis is embarking on the task of implementing enhancements to an already established platform, which is recognized for its focus on the interoperability features of IoT ecosystems. The enhancements involve the integration of Apache Kafka\cite{WhatIsApacheKafka}, a continuously evolving technology, along with its associated technologies. Our objective is to expand and upgrade the SYNAISTHISI platform, by incorporating an advanced system that propels the process of data transfer and flow to the next level. Apache Kafka, with its event streaming capabilities across various systems, leads the way for fast and responsive IoT applications. Its high throughput, high availability and low latency make it an ideal choice (cf. Sec. \hyperref[event_streaming]{3.2}). With the proper configuration and infrastructure, Kafka can transform small-scale applications into large-scale projects. By incorporating new components, primarily from the Apache project, we can seamlessly interconnect existing protocols and devices with newer ones in an easy and accessible manner. We persist in the fundamental approach of upgradability and expandability, thus ensuring the creation of an IoT platform capable of hosting multiple services and ensuring compatibility with future services across a broad range of use cases.

\clearpage
\section{Contributions}

The contributions of this thesis are summarized as follows:
\begin{enumerate}
\item Design and creation of a comprehensive cloud-based IoT Air Quality monitoring application.
\item Enhanced the SYNAISTHISI platform's event-driven capabilities through the integration of Apache Kafka, Kafka Connect and Schema Registry.
\item Implemented Kafka Connect to interconnect events across multiple protocols, and employed Schema Registry for efficient data validation and message optimization.
\item Develop and configure a high-performance, fault-tolerant and scalable Apache Kafka cluster.
\item Implemented a user-friendly and contemporary web application for real-time air quality data consumption on a live map, utilizing Web-Sockets and React.
\end{enumerate}


\section{Thesis Organization}

The subsequent sections of this thesis are organized as follows: Background, motivation and related work are provided in Chapter 2 (cf. Chap.\hyperref[chap2]{2}), setting the theoretical approach. A detailed analysis of the utilized frameworks and services is presented in Chapter 3 (cf. Chap.\hyperref[chap3]{3}). A comprehensive description of the application's architecture, development process and implementation, including the design of our front-end map, is offered in Chapter 4 (cf. Chap.\hyperref[chap4]{4}). Experimental evaluation and bench-marking of the application, providing an in-depth performance analysis, are the focus of Chapter 5 (cf. Chap.\hyperref[chap5]{5}). Finally, potential future work and expansions, which concludes the thesis with reflections and directions for further research are discussed in Chapter 6 (cf. Chap.\hyperref[chap6]{6}).