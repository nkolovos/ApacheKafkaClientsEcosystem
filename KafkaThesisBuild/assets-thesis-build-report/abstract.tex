% Ελληνική Περίληψη

\begin{abstract}\setlanguage{greek}
Σε αυτή την εργασία αναπτύσσουμε μια σύνθετη εφαρμογή για το Διαδίκτυο των Πραγμάτων(IoT), όπου καταγράφουμε τα επίπεδα συγκέντρωσης PM\textsubscript{2.5} σωματιδίων, την υγρασία και τη θερμοκρασία. Η λύση μας ,αξιοποιεί αναπτυγμένες εργαλεία cloud και επεκτείνει την πλατφόρμα «SYNAISTHISI».

Η ατμοσφαιρική ρύπανση αποτελεί μια σημαντική ανησυχία για πολλούς ανθρώπους τα τελευταία χρόνια. Σκοπός μας είναι να προσφέρουμε μια λύση φιλική προς τον χρήστη και για όποιον επιθυμεί να παρακολουθεί την ατμοσφαιρική ποιότητα σε κάποια περιοχή. Κάθε πολίτης θα πρέπει να έχει το δικαίωμα να γνωρίζει την ποιότητα της ατμόσφαιρας όπου διαμένει και επιπλέον να είναι σε θέση να δημιουργήσει ακόμα και τα δικά του εργαλεία καταγραφής με εύκολο και προσιτό τρόπο. Βασισμένοι σε αυτό, δημιουργήσαμε μια πλήρως αξιοποιήσιμη και επεκτάσιμη εφαρμογή, αξιοποιώντας αναπτυγμένα cloud εργαλεία και τεχνολογίες, σε συνδυασμό με εγκαθιδρυμένα πρωτόκολλα του Διαδικτύου των Πραγμάτων(IoT).

Τα τελευταία χρόνια, έχουν υπάρξει πολλές προσπάθειες στη δημιουργία εφαρμογών του Διαδικτύου των Πραγμάτων, που βοηθούν στην επίλυση πολλών προβλημάτων (στην περίπτωση μας ατμοσφαιρικη ρύπανση), χρησιμοποιώντας απλές και προσιτές υλοποιήσεις. Αυτή τη φορά προσπαθούμε να πάμε ένα βήμα παρακάτω και να αναπτύξουμε μια σύνθετη στη δημιουργία της, αλλά ξεκάθαρη κατά τη χρήση της, cloud εφαρμογή. Ακολούθουμε τις βασικές αρχές των περισσότερων εφαρμογών του Διαδικτύου των Πραγμάτων, κάνοντας την υλοποίηση μας προσιτή και βολική προς του χρήστες της. 

Το βασικό εργαλείο που αξιοποιούμαι και αναπτύσσουμε είναι το Apache Kafka, ένα ισχυρό εργαλείο που προσφέρει  υψηλή ταχύτητα επεξεργασίας, χαμηλή απόκριση και αυξημένη διαθεσιμότητα στην εφαρμογή μας. Η δομή της υλοποίησης σκοπεύει στην επέκταση της υπάρχουσας πλατφόρμας «SYNAISTHISI», αξιοποιώντας νέα εργαλεία τα οποία μας επιτρέπουν να διασυνδέσουμε υπάρχοντα πρωτόκολλα επικοινωνίας (MQTT, RabbitMQ), με τις σύνθετες δυνατότητες του Apache Kafka και του οικοσυστήματος του.

Σκοπός μας είναι η καταγραφή της ποιότητας της ατμόσφαιρας και η διανομή αυτών των δεδομένων σε οποιοδήποτε από τα υποσυστήματα της πλατφόρμας «SYNAISTHISI». Η προσθήκη του Apache Kafka και των παρεμφερών εφαρμογών του μας δίνει τη δυνατότητα να φιλοξενήσουμε στην πλατφόρμα ακόμα πιο συνθέτες cloud εφαρμογές. Συνεπώς αναπτύσσοντας την εφαρμογή για την ατμοσφαιρική καταγραφή, έχουμε τη δυνατότητα να αναλύσουμε σε βάθος τις τεχνολογίες που αξιοποιήσαμε κατά τη υλοποίηση της. Αυτές οι τεχνολογίες μπορούν να χρησιμοποιηθούν και σε άλλες εφαρμογές, ακολουθώντας το βασικό όραμα της πλατφόρμας «SYNAISTHISI».


\vspace*{\fill}
\noindent{\large\bf{Λέξεις-κλειδιά:}}\\ 
Διαδίκτυο των Πραγμάτων, PM\textsubscript{2.5} σωματίδια,  SYNAISTHISI, cloud, ατμοσφαιρική ρύπανση, Apache Kafka
\end{abstract}







